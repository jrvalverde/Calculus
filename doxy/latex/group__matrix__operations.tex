\section{Matrix\_\-operations}
\label{group__matrix__operations}\index{Matrix_operations@{Matrix\_\-operations}}
\subsection*{Functions}
\begin{CompactItemize}
\item 
public {\bf status} {\bf mat\_\-assign} ({\bf matrix} dest, {\bf matrix} orig)
\begin{CompactList}\small\item\em MAT dest = orig.\item\end{CompactList}\item 
public {\bf status} {\bf mat\_\-transpose} ({\bf matrix} trn, {\bf matrix} mat)
\begin{CompactList}\small\item\em MAT trn = TRN(mat) Compute the transpose of a matrix.\item\end{CompactList}\item 
public {\bf status} {\bf mat\_\-invert} ({\bf matrix} B, {\bf matrix} C)
\begin{CompactList}\small\item\em MAT B = INV ( C );.\item\end{CompactList}\end{CompactItemize}


\subsection{Function Documentation}
\index{Matrix_operations@{Matrix\_\-operations}!mat_assign@{mat\_\-assign}}
\index{mat_assign@{mat\_\-assign}!Matrix_operations@{Matrix\_\-operations}}
\subsubsection{\setlength{\rightskip}{0pt plus 5cm}public {\bf status} mat\_\-assign ({\bf matrix} {\em dest}, {\bf matrix} {\em orig})}\label{group__matrix__operations_a0}


MAT dest = orig.

Transfer ([m] x [n]) elements from 2 to 1.

Assigns the values of one matrix to another. Both matrices must have the same dimensions, otherwise an error result will occur.

If the origin matrix is NULL, then the destination matrix will be set to the null matrix (i.e. all of its elements initialized to zero).\begin{Desc}
\item[Parameters: ]\par
\begin{description}
\item[{\em 
dest}]the destination matrix \item[{\em 
orig}]the origin matrix\end{description}
\end{Desc}
\begin{Desc}
\item[Returns: ]\par
SUCCESS if all went well, an error code otherwise \end{Desc}


Definition at line 84 of file mat\_\-ops.c.

References matrix::cols, MAT\_\-BOUNDSCHECK, mat\_\-init(), real, matrix::rows, status, SUCCESS, and matrix::values.

Referenced by mat\_\-invert().\index{Matrix_operations@{Matrix\_\-operations}!mat_invert@{mat\_\-invert}}
\index{mat_invert@{mat\_\-invert}!Matrix_operations@{Matrix\_\-operations}}
\subsubsection{\setlength{\rightskip}{0pt plus 5cm}public {\bf status} mat\_\-invert ({\bf matrix} {\em B}, {\bf matrix} {\em C})}\label{group__matrix__operations_a2}


MAT B = INV ( C );.

Computes the inverted matrix of C, which is a square matrix n x n and stores the result on B. The algorithm employed is a modification of Gauss' method. This is probably not the best (nor is it the nicest implementation), but for the moment will do.

Given the identity matrix I$_{\mbox{nn}}$, we may think of another such that

A B = B A = I =$>$ B = A$^{\mbox{-1}}$

A A$^{\mbox{-1}}$ = I = A$^{\mbox{-1}}$ A

Using the cofactors of A$_{\mbox{ij}}$ we may compute

A$^{\mbox{-1}}$ = 1 / $|$A$|$ 183 adj(A)

(calculating through the adjunct matrix). If $|$A$|$ = 0 then it is not defined and A is SINGULAR.

The inverse of a matrix A, if it exists, is unique and may be found: We first form for A$_{\mbox{nn}}$ the matrix A$_{\mbox{n x 2n}}$ \[ (A, I) = \pmatrix { a_{11} & a_{12} & \cdots & a_{1n} & 1 & 0 & \cdots & 0 \cr a_{21} & a_{22} & \cdots & a_{2n} & 0 & 1 & \cdots & 0 \cr \vdots & \vdots & \ddots & \vdots & \vdots & \vdots & \ddots & 0 \cr a_{n1} & a_{n2} & \cdots & a_{nn} & 0 & 0 & \cdots & 1 \cr } \] That is, the left half is A and the right half I, the identity matrix. Using a modified Gauss method we transform the former matrix into \[ (I, B) = \pmatrix { 1 & 0 & \cdots & 0 & b_{11} & b_{12} & \cdots & b_{1n}\cr 0 & 1 & \cdots & 0 & b_{21} & b_{22} & \cdots & b_{2n}\cr \vdots & \vdots & \ddots & 0 & \vdots & \vdots & \ddots & \vdots\cr 0 & 0 & \cdots & 1 & b_{n1} & b_{n2} & \cdots & b_{nn}\cr } \] Now the left half is I and the right half, B is the inverse of A.

To avoid changing matrix C we get first a working copy that we store in matrix A, whose space is reserved ex-profeso and is freed upon termination.

@callgraph 

Definition at line 224 of file mat\_\-ops.c.

References FALSE, mat\_\-alloc(), mat\_\-assign(), mat\_\-free(), mat\_\-identity(), MAT\_\-NOMEMORY, MAT\_\-SINGULAR, real, matrix::rows, status, TRUE, and matrix::values.\index{Matrix_operations@{Matrix\_\-operations}!mat_transpose@{mat\_\-transpose}}
\index{mat_transpose@{mat\_\-transpose}!Matrix_operations@{Matrix\_\-operations}}
\subsubsection{\setlength{\rightskip}{0pt plus 5cm}public {\bf status} mat\_\-transpose ({\bf matrix} {\em trn}, {\bf matrix} {\em mat})}\label{group__matrix__operations_a1}


MAT trn = TRN(mat) Compute the transpose of a matrix.

The transpose A$^{\mbox{T}}$ of a matrix A is A' = A$^{\mbox{T}}$ / row A' = col A =$>$

(A)$_{\mbox{ij}}$ = (A$^{\mbox{T}}$)$_{\mbox{ji}}$

Thus, if A$_{\mbox{m183n}}$, then A'$_{\mbox{n.m}}$ =$>$

If A 183 B = C -$>$ C' = A' 183 B'

We may therefore describe a column vector as a row vector

\[ \pmatrix { x_1 \cr x_2 \cr \cdots \cr x_{n-1} \cr x_n \cr } = \pmatrix {x_1 & x_2 & \cdots & x_n \cr } \]\begin{Desc}
\item[Parameters: ]\par
\begin{description}
\item[{\em 
trn}]a matrix where we will store the transpose \item[{\em 
mat}]the matrix whose transpose we want to take \end{description}
\end{Desc}


Definition at line 147 of file mat\_\-ops.c.

References matrix::cols, MAT\_\-BOUNDSCHECK, real, matrix::rows, status, SUCCESS, and matrix::values.