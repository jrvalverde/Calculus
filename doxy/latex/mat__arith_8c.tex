\section{mat\_\-arith.c File Reference}
\label{mat__arith_8c}\index{mat_arith.c@{mat\_\-arith.c}}
Basic arithmetic operations with matrices. 


{\tt \#include $<$stdio.h$>$}\par
{\tt \#include $<$math.h$>$}\par
{\tt \#include $<$stdlib.h$>$}\par
{\tt \#include $<$bits/nan.h$>$}\par
{\tt \#include \char`\"{}portable.h\char`\"{}}\par
{\tt \#include \char`\"{}matrix.h\char`\"{}}\par


Include dependency graph for mat\_\-arith.c:\subsection*{Functions}
\begin{CompactItemize}
\item 
public {\bf status} {\bf mat\_\-sum} ({\bf matrix} result, {\bf matrix} mat1, {\bf matrix} mat2)
\begin{CompactList}\small\item\em MAT C = A + B;.\item\end{CompactList}\item 
public {\bf status} {\bf mat\_\-substract} ({\bf matrix} result, {\bf matrix} mat1, {\bf matrix} mat2)
\begin{CompactList}\small\item\em MAT res = mat1 - mat2.\item\end{CompactList}\item 
public {\bf status} {\bf mat\_\-multiply} ({\bf matrix} prod, {\bf matrix} fact1, {\bf matrix} fact2)
\begin{CompactList}\small\item\em MAT prod[m][p] = fact1[m][n] $\ast$ fact2[n][p]; Compute the product of two matrices.\item\end{CompactList}\item 
public {\bf status} {\bf mat\_\-scalar\_\-product} ({\bf matrix} scprod, {\bf matrix} mat, {\bf real} lambda)
\begin{CompactList}\small\item\em compute the scalar product\item\end{CompactList}\end{CompactItemize}


\subsection{Detailed Description}
Basic arithmetic operations with matrices.



An implementation of marix arithmetics. This module includes functions to perform matrix summation, substraction, multiplication and computing the scalar product. These operations are base to more advanced matrix calculus manipulations.

\begin{Desc}
\item[Precondition: ]\par
{\bf portable.h}\end{Desc}
\begin{Desc}
\item[Precondition: ]\par
{\bf matrix.h}\end{Desc}
\begin{Desc}
\item[See also: ]\par
{\bf matrix.h} for a general introduction to matrices\end{Desc}
\begin{Desc}
\item[See also: ]\par
{\bf mat\_\-init.c} to learn how you can assign initial values to a matrix or its elements.\end{Desc}
\begin{Desc}
\item[See also: ]\par
{\bf mat\_\-ops.c} to learn more about how to perform basic matrix operations.\end{Desc}
\begin{Desc}
\item[See also: ]\par
{\bf mat\_\-test.c} Is an auxiliary module with the test code of these functions used to debug this module (last time everything went OK).\end{Desc}
\begin{Desc}
\item[Author: ]\par
Jos\'{e} Ram\'{o}n Valverde Carrillo ({\tt jrvalverde@acm.org})\end{Desc}
\begin{Desc}
\item[Version: ]\par
3.0\end{Desc}
\begin{Desc}
\item[Date: ]\par
23 - february - 2004 v3.0\end{Desc}
\begin{Desc}
\item[Date: ]\par
11 - february - 2004 v2.0\end{Desc}
\begin{Desc}
\item[Date: ]\par
1 - october - 1988 Last modification of v1.0\end{Desc}
COPYRIGHT: 169 Yo\-Ego. Since I have no cash, I can't register this (nor do I believe I should). So this module is left in the PUBLIC DOMAIN. It is furthermore forbidden its use for commercial purposes unless I get a share on the profits. I say. Yo\-Ego.

\$Id\$ \$Log\$



Definition in file {\bf mat\_\-arith.c}.